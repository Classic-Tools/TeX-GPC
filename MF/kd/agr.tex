\font\agr=kdgr10
\endinput

AGR ist eine 10 Punkt Schrift fuer altgriechisch.  Sie entstammt
einer Fontfamilie  von K J Dryllerakis, die ihrerseits auf Arbeiten
von Sylvio Levi beruht.

August 2008, Wolfgang Helbig


Einige Zeichen mit vielleicht ueberraschender ASCII-Entsprechung:
griechisch	ASCII
Eta		h
Theta 		th
Sigma 		s (auch am Wortende, TeX setzt dann das andere Schluss-Sigma)
Ypsilon		u
Phi		ph 
Chi		kh
Psi		ps
Omega 		w

Da TeX th als Theta sieht, schreibe t{}h fuer Tau Eta.
Analog fuer die anderen beiden Aspiratae.

Die Ornamente ueber oder unter einem Vokal erwartet TeX in der Reihenfolge
Hauchzeichen oder Trema,  Akzent, Vokal, iota subcriptum.

Erste Gruppe Hauchzeichen:  
spiritus asper   <	(auch vor Rho moeglich)
spiritus lenis   >
Trema            "	(nur vor Iota oder Ypsilon)

Zweite Gruppe Akzente:
Akut 		 '
Gravis 	   	 `
Zirkumflex       '`

Dritte Gruppe iota subscriptum:
iota subscriptum |	

Weitere Zeichen:
Apostroph am Wortende (Ellision)	'
oeffnende Anfuehrungszeichen		((
schliessende Anfuehrungszeichen	))
Hochpunkt ("Kolon")			;
Semikolon 				?

Der Rest ist so, wie man denkt.
